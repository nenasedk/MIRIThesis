\chapter{Atmospheric Retrievals}
Every photon of light that we receive from an exoplanet will interact with its atmosphere, and will therefore provide us with a hint of what that atmosphere may look like.
An atmospheric retrieval is the process of reconstructing the atmosphere of an object based on an observed spectrum.
This process relies heavily on having accurate models which can be parameterized by the physical quantities we are interested in: generally the temperature, pressure and composition \parencite{Madhusudhan2018}.
As these models cover a very large parameter space (>10 parameters, each covering several orders of magnitude), it is necessary to have an efficient method for sampling this space, computing a model and comparing this model to the data 
\parencite{Benneke2012,Benneke2013}.
Currently, atmospheric retrieval methods have been used for both exoplanets and brown dwarfs to identify water, methane, CO, CO$_{2}$ and other species \parencite{Konopacky2013,Barman2015}, along with clouds being a universal feature \parencite{Line2017,Schlawin2018,Morley2018}

This chapter will outline the process of an atmospheric retrieval from modeling to marginalization of posteriors, and will examine the impact that the instrumental effects described in Chapter \ref{ch:fringe} have on the retrieved parameters. 
Additionally, this will provide an example of how the MIRI MRS can be used to explore exoplanet and brown dwarf atmospheres, and what observational parameters should be considered when studying these objects, following similar studies from \parencite{Batalha2018,Schlawin2018} for NIRSpec and LRS observations and \parencite{Feng2018} for future reflected light missions.


 %atmo ret,basically everything nested samp, 
%\cite{Fisher2019} %Cross corr machine learning hoeijmakers
%\cite{Oreshenko2019}% Model grid comparison random forest
%\cite{Barman2015} %HR8799b water methan CO 

%\cite{Blanco-Cuaresma2018} %Stel spec caveats
%\cite{Konopacky2013} %CO and water in hr8799

%\parencite{Morley2018} %D/H ratios
%\cite{Lupu2016} %Is actually 2016 - reflected light atmo rets (multinest)
%\cite{Gandhi2018} %HyDRA retrieval code
%\cite{Baudino2017} %Troublesome model params
%\cite{Line2013} %Secondary eclipse retrieval technique comparison
%\cite{Madhusudhan2018b} % With Seagar - temp nad abundance retrieval pt profile
%\cite{Irwin2008} %Nemisis atmo ret code
%\cite{Robinson2016} %Observations of planets with coronographs in space %(WFIRST)
%\cite{Waldmann2015} %TauRex
%\cite{Waldmann2015a} %TauRex emission
%\cite{Line2015,Line2017,Zalesky2019} % Brown dwarf retrieval part 1,2,3

%\cite{Batalha2018}%MIRI LRS - strategies (obs)
% Gravity Beta Pic - petitRadTrans retrieval how to, CO
%\cite{Feng2018} %Future ref light: basis of mine but emission
%\cite{Molliere2019}% Two papers, isotopologues and petitRadTrans
\subsubsection{Atmospheric Modeling}
Atmospheric modeling is the task of creating an spectra based on the physical properties of the atmosphere.
This is a broad task that can range from a 3D Global Circulation Model (GCM) which accounts for self-consistent atmospheric chemistry \parencite{Chen2019} to a 1D model based around an empirical temperature-pressure profile \parencite{Molliere2019}.
The choice of model depends largely on the requirements for accuracy and computational cost. 
Considering the potentially millions of possible atmospheres that must be examined in a retrieval problem, whatever model is used must be computationally efficient while still maintaining a close approximation of reality.

%\cite{Zhang2019} %PLATON Transmission spectra generation
\section{petitRADTRANS}
For this work we chose to use the petitRADTRANS package due to its user-friendly python implementation, high speed computation for retrieval use and extensive high-resolution, line-by-line spectral library for generating planetary spectra \parencite{Molliere2019}. 
It is a 1D, radiative transfer package with many parameters options, described in table \ref{tab:petitradparams}
PetitRADTRANS can compute both emission and transmission spectra, with an output spectral resolution of R=1000 in correlated k mode, or R=1 000 000 in line-by-line mode. 

\begin{table}[t]
	\centering
	\begin{tabular}{ll}
		\toprule
		\textbf{Property} & \textbf{Description}\\
		\midrule
		Temperature & Parameterized, e.g. \parencite{Guillot2010}\\
		Abundances & Parameterized, e.g. vertically constant\\
		Scattering & Cloud scattering, transmission spectra only\\
		Clouds & Power law and condensation clouds\\
		Cloud particle size & $f_{SED}$ and K$_{ZZ}$ or parameterized\\
		Particle size distribution & log-normal, variable width\\
		Cloud abundance & Parameterized\\
		Wavelength spacing & R=1000 (c-k), 10$^{6}$ (lbl)\\
		Valid emission spectra & Clear, from NIR and longer\\
		\bottomrule
	\end{tabular}
	\caption{Description of the parameters available in petitRADTRANS. For cloud particles, $f_{SED}$ is the mass-averaged ratio of the cloud particle settling speed and mixing velocity. K$_{ZZ}$ is the atmospheric eddy diffusion coefficient \parencite{Ackerman2001}}
	\label{tab:petitradparams}
\end{table}

Note that much of the following sections applies to many other similar 1D radiative transfer atmospheric modeling programs such as ATMO \parencite{Goyal2019}, Planetary Spectrum Generator \parencite{Villanueva2018}, HELIOS \parencite{Malik2017,Malik2019} and others.
Many (or even most) of these programs rely on the same set of high-resolution molecular line lists, including HITRAN/HITEMP \parencite{Rothman1981,Rothman2010,Gordon2017}, ExoMol/ExoCross \parencite{Tennyson2016,Tennyson2017,Yurchenko2018} and others. 
%\cite{Behmard2019}%Cool star spectroscopy, hires

\subsection{Radiative Transfer}
In order to compute the emission spectrum an initial featureless blackbody spectrum $B(T_{int})$ is passed through multiple discrete layers of the atmosphere, parameterized by their temperature, pressure, and the opacities of each of the species present in a given layer.
Modeling each layer as plane parallel, the intensity is computed as in \parencite{Irwin2008,Molliere2017,Molliere2019}
\begin{equation}
I_{top} = B(T_{int})\mathcal{T}^{atmo} + \frac{1}{2}\sum_{i=0}^{N_{L}-1}\left[B(T^{i}) + B(T^{i+1})\right]\left(\mathcal{T}^{i}-\mathcal{T}^{i+1}\right).
\end{equation}
$N_{L}$ is the number of layers in the atmosphere, and $\mathcal{T}$ is the transmission from a given layer to the top of the atmosphere. All quantities are averaged per wavelength bin in c-k mode, while they are evaluated at each wavelength point in line-by-line mode.

In order to compute the temperature structure of the atmosphere, a modified Guillot model \parencite{Guillot2010} was used, as in \parencite{Molliere2017,Molliere2019}.
The temperature structure is defined as 
\begin{equation}
T(P) = \left<T_{Guillot}(P)\left(1-\frac{\alpha}{1+P/P_{trans}}\right)\right>_{P}.
\end{equation}
As denoted by $\left<\right>_{P}$, this profile is boxcar-smoothed over log(P), with bin widths of 1.25 dex \parencite{Molliere2019}. 
The modified Guillot profile is defined as
\begin{align}
T_{Guillot}(P) &= \frac{3T^{4}_{int}}{4}\left(\frac{2}{3} + \delta P\right)\\
&+\frac{3T^{4}_{eq}}{4}\left[\frac{2}{3} + \frac{1}{\gamma\sqrt{3}} + \left(\frac{\gamma}{\sqrt{3}} - \frac{1}{\gamma\sqrt{3}}\right)e^{-\gamma\delta\sqrt{3}P}\right],
\end{align}
with the second term accounting for irradiation of the upper atmosphere.
The opacity parameters $\delta=\kappa_{IR}/g$ and  $\gamma=\kappa_{V}/\kappa_{IR}$ are defined such that the optical depth $\tau=\delta\cdot P$. 
$T_{int}$ is the internal temperature of the planet, and is equivalent to the effective temperature of the planet, that is the temperature of a blackbody with the same luminosity as itself.
$T_{eq}$ is the equilibrium temperature of the planet, based on the temperature of the host star and separation of the planet. 
For an isolated, free floating object this temperature goes to 0.
The remaining free parameters $\alpha$, $P_{trans}$ simply modify the shape of the temperature structure according to the pressure.

\subsection{Opacity Sources}
To compute the emission spectra of an atmosphere, petitRADTRANS accounts for various opacity contributions including absorption and emission lines, collisionally induced absorption, cloud opacity and scattering and Rayleigh scattering cross sections. These sources are described in detail in \parencite{Molliere2019}, summarized in tables 2 and 3. For this work we consider only the case of a cloud-free atmosphere due to the complexity of realistic cloud modeling.
\subsubsection{Line-by-line}
In its high resolution line-by-line mode, petitRADTRANS computes emission spectra with R=10$^{6}$. 
These spectra are computed using opacity sources for molecular and atomic lines from ExoMol/ExoCross library \parencite{Yurchenko2018}. Pressure broadening is taken into account using the coefficients from HITRAN/HITEMP \parencite{Rothman2010,Rothman2013} or from \parencite{Sharp2007} (Eqn. 15). The line opacities are computed from 80-3000 K, and from 0.3-28 $\mu$m in high resolution mode.
\subsubsection{Correlated K}
The low resolution mode of petitRADTRANS uses the correlated-k (c-k) method of computing line opacities \parencite{Goody1989,Lacis1991,Fu1992}. 
This method for calculating emission and absorption features assumes that the opacity distribution functions between differing species are uncorrelated, which permits simple computation of overlapping features. 
While petitRADTRANS implements a c-k method with a spectral resolution of 1000, in principle it is accurate to much higher resolutions.
However, the principle utility of the c-k method is in the dramatic reduction in computational cost for computing a spectra such that petitRADTRANS can be used as the foundation for an atmospheric retrieval code requiring hundreds of thousands or millions of models to be generated. 
\parencite{Molliere2019} discusses the variations between the results of the line-by-line method and the c-k method, finding discrepancy of at most 6\%.
Typical variation is much lower, as seen in Fig. 2 of \parencite{Molliere2019}.

\subsubsection{Clouds}
While clouds are a seemingly universal feature in exoplanets and brown dwarfs \parencite{Morley2014,Line2016,Faherty2018}, they remain a difficult problem for retrieval studies. 
Clouds form when a species condenses out of the gas phase, typically around a small nucleus.
This creates a layer of particles at a reasonably well defined altitude in the atmosphere, and prevents the observation of deeper atmospheric layers.
While a simple model of clouds may be a `gray' cloud deck that acts uniformly across wavelength, a more complex model will account for differing IR and visible opacities, as well as particle scattering and other complex microphysics.
From experience on Earth and within the solar system, cloud systems are highly complex and variable, with shifting cloud coverage and structure.
The mid infrared in particular may allow for an observational window to probe deeper atmospheric layers and begin to characterize cloud composition.
In current retrieval codes, clouds are generally designed from a simple model based on a given particle distribution \parencite{Ackerman2001}, or simply as a gray cloud deck at a specified pressure level.
Both of these models are implemented in petitRADTRANS.
These models do not agree well with microphysical models, and lead to substantial difficulties in the interpretation of retrieved spectra.
This remains an open problem for atmospheric retrievals, and we do not attempt to examine cloud effects in this work, instead choosing the simple, though unrealistic case of a clear atmosphere. 
\parencite{Schlawin2018} examines potential impacts of clouds on atmospheric retrievals with the MIRI LRS.

\section{Bayesian Inference}
An atmospheric retrieval is the process of extracting information about physical parameters from a measured spectrum. 
In general this procedure involves comparing the data to a series of template spectra with known parameters and identifying the best fit model.
Unfortunately for astronomers, atmospheres are complicated: typical one 1D models still require many (>15) parameters to generate a somewhat realistic model. 
This results in a very large parameter space in which to search for the correct set of properties that describe our measurement.

Monte Carlo methods, including Nested Sampling, are used to effectively search this large space using the Baysian evidence as a goodness-of-fit metric.
Here we will follow \parencite{Speagle2019} to provide a brief overview of Bayesian inference.

To measure the likelihood of a given model, we turn to Bayes' Theorem:
\begin{equation}\label{eqn:bayes}
P(\mathbf{\Theta}_{M}|\mathbf{D},M) = \frac{P(\mathbf{D}|\mathbf{\Theta},M)P(\mathbf{\Theta}|M)}{P(\mathbf{D}|M)}.
\end{equation}
In our notation, $\mathbf{\Theta}$ is the set of parameters that describe a model $M$, that is fit to the data $\mathbf{D}$. 
Bayes' theorem asks what is the probability that the parameters $\mathbf{\Theta}$ are true given the data and model. 
The distributions for each parameter are the \textbf{posterior} distributions.

This is then related to the \textbf{likelihood} $P(\mathbf{D}|\mathbf{\Theta},M)$ of measuring the data given the model, the \textbf{prior} probability $P(\mathbf{\Theta}|M)$ which describes our degree of belief in our model and the \textbf{evidence} $P(\mathbf{D}|M)$, which is marginalized over all possible $\mathbf{\Theta}$ and quantifies how well the model describes the data.
To simplify notation, we adopt the following convention for Bayes' theorem:
\begin{equation}
\mathcal{P}(\mathbf{\Theta}) = \frac{\mathcal{L}(\mathbf{\Theta})\pi(\mathbf{\Theta})}{\mathcal{Z}}.
\end{equation}

In general, the goal of an atmospheric retrieval is to find the best fit model by maximizing the evidence $\mathcal{Z}$, and as a by product finding the marginalized posterior distributions for each parameter.
This comes with many challenges, especially when dealing with large numbers of parameters.
Selection of the priors and model will determine the extent to which a result can be interpreted, while sampling large parameter spaces and computing likelihoods introduces substantial numerical challenges. 
The Nested Sampling method described below attempts to solve the challenges of exploring a large parameter space.
%\subsection{MCMC}
%An MCMC method generates many 
%\cite{Foreman-Mackey2013} %emcee
%\cite{MacKay2003} %MCMC

\subsection{Nested Sampling}
Nested sampling attempts to address several of the shortcomings of Markov Chain Monte Carlo (MCMC) methods while simultaneously improving computational efficiency \parencite{Skilling2004}.
A particular strength of the method is in the sampling of highly multimodal distributions, removing the problem where an MCMC approach may get stuck in a single local maximum.
MCMC methods generate samples `proportional to' the true posterior distributions, which lead to difficulties in computing the evidence $\mathcal{Z}$ \parencite{Speagle2020}. 
In contrast, nested sampling puts the evidence first and provides estimates of the posterior distributions from the importance weights of the final set of samples. First described in \parencite{Skilling2004}, nested sampling has been adopted as the sampling algorithm of choice within the astrophysics community \parencite{Feroz2009,Buchner2014,Feroz2019,Speagle2020}.

With the goal of parameter estimation, nested sampling attempts to estimate the evidence $\mathcal{Z}$ rather than directly sampling the posteriors \parencite{Skilling2004}. 
This is done by integrating over the entire parameter space of $\mathbf{\Theta}$
\begin{equation}
\mathcal{Z} = \int_{\Omega_{\mathbf{\Theta}}}\mathcal{L}(\mathbf{\Theta})\pi(\mathbf{\Theta})d\mathbf{\Theta}.
\end{equation}
This is difficult.

Rather than attempting to directly solve the entire multidimensional integral, nested sampling transforms this into an integration over the \textit{prior} volume X:
\begin{equation}
\mathcal{Z} = \int_{\Omega_{\mathbf{\Theta}}}\mathcal{L}(\mathbf{\Theta})\pi(\mathbf{\Theta})d\mathbf{\Theta} = \int_{0}^{1}\mathcal{L}(X)dX.
\end{equation}
This is now a contour integral over isocontours $\mathcal{L}(X)$ which bound the prior volume
\begin{equation}
X(\lambda) = \int_{\Omega_{\mathbf{\Theta}}:\mathcal{L}(\mathbf{\Theta})\geq\lambda}\pi(\mathbf{\Theta})d\mathbf{\Theta},
\end{equation} 
which is the fraction of the prior where the likelihood of the data given the model is above some threshold $\lambda$.
The integration is now simplified into a 1D integration over X, given proper prior selection.

\subsubsection{Method}
Consider a parameter space with $D$ dimensions.
We will describe this space as a unit hypercube, where each parameter runs from 0 to 1.
Priors are thus transformations from this space to a physical parameter space.
Often the prior is a uniform distribution, which simply scales the space, but it may also be an informative prior such as a normal distribution centered at an expected physical value.
In order to sample this space, $N_{L}$ `live points' are generated, each of which provides a set of parameters $\mathbf{\Theta}$. 
$N_{L}$ must be greater than $D+1$, and typically values on the order of $50\times D$ are used \parencite{Feroz2009}.
Using a likelihood function $\mathcal{L}(\mathbf{\Theta})$, the evidence $\mathcal{Z}$ can be computed by comparing the model to the data.
Having computed the evidence at each point, the live points are then sorted and the point with the lowest evidence is discarded.
A set of ellipsoids is drawn around the remaining points. 
The procedure for computing these ellipsoids is given in \parencite{Feroz2008,Feroz2009}.
By using a set of ellipsoids, multiple modes in the parameter space can be encompassed.
Once the ellipsoids bounding the remaining points are drawn, a new sample is drawn from within the restricted sample space.
The evidence for the new point is computed, and it is accepted if the evidence is greater than the minimum evidence of the previous remaining set of points.
The entire procedure is repeated until some convergence criteria is satisfied, with each iteration resulting in a smaller volume being encompassed by the ellipsoids, nested within the previous volume.

This procedure can be improved in many ways, including importance nested sampling \parencite{Feroz2019} and dyamic nested sampling \parencite{Speagle2020}. 

\subsection{Multinest}
For our implementation of an atmospheric retrieval code, we chose to use the Multinest algorithm \parencite{Feroz2009} using the pyMultinest wrapper \parencite{Buchner2014} and using importance nested sampling to improve the accuracy of the Bayesian evidence calculation \parencite{Feroz2019}.
This particular implementation of nested sampling is commonly used in atmospheric retrieval codes due to its fast Fortran implementation, though it was initially developed for cosmological problems.

Using the pyMultinest package, we implemented the required log-prior function which transforms the unit hypercube to physical parameter space and the log-likelihood function used to compare the model to the data. The full code is available at \url{https://github.com/nenasedk/petitRetrieval}, and is based of the emission spectrum retrieval described in \parencite{Molliere2019}. 
Retrievals were typically performed using 500 or 1000 live points, with the convergence criteria 
\begin{equation}
\Delta\ln\mathcal{Z} = \ln{Z_{i} - Z_{i+1}}
\end{equation}
set to 0.3 for parameter estimation and 0.8 for model comparison, as suggested in the pyMultinest documentation.

\section{Observations}\label{sec:obs}
The targets used in our retrieval study are guided by the JWST ERS and GTO programs. This allows us to use well-defined observing strategies for each object, and present a clear case for the science that can be accomplished with these observations.
While all three were discussed in Chapter 1, we will now outline the proposed observing strategies and science cases for each target.
\subsubsection{VHS-1256B}
VHS-1256b is a young (0.2Gyr), high mass (11.2M$_{Jup}$) planet at a distance of 12.7pc \parencite{Bowler2016}. 
The wide separation of 8" makes it an easy target for observation with the MRS, as its host star will fall outside of the FoV.
It has a J-band magnitude of 16.662, and a late L spectral type \parencite{Miles2018}.
As an object of interest for the JWST ERS program 1386, it will be observed with the NIRCam imager, along with both the NIRSpec and MRS spectrometers \parencite{Hinkley2019}.
Using the MRS, VHS-1256b will be observed using a SLOW readout pattern, using 30 groups per integration, with one integration per exposure using a 2 point dither pattern.
This results in a total exposure time of 1433.395s in each of the MRS sub-bands, and will cover the full wavelength range of the MRS.
It will be simultaneously imaged using the MIRIM instrument.
An additional background only exposure will be taken using the same exposure parameters, but without dithering, for a total of half of the science exposure time.

Methane spectral features have been detected in the L-band spectrum of VHS-1256b \parencite{Miles2018}, but mid infrared spectroscopy will allow the use of methane and other molecules to characterize atmospheric properties such as dis-equilibrium chemistry and vertical mixing \parencite{Beichman2019}. 
\subsubsection{2M1207b}
2M1207b is a 1600K, 10 M$_{Jup}$ object at wide separation from its brown dwarf primary (TWA 27) and a distance of 52.4pc \parencite{Bowler2016}.
In comparison to VHS-1256b, 2M1207b has a relatively small separation of 0.77", which is more characteristic of currently known objects.
As one of the first directly imaged exoplanets, it provides a template for characterizing young, hot objects, and will be observed in the JWST GTO program 1270 \parencite{Birkmann2019}.
This observation will use the NIRSpec IFU, MIRIM and the MIRI MRS.

Using the MRS, 2M1207b will be observed using a FAST readout to prevent detector saturation, using 76 groups per integration, and one integration per exposure. 
It will use a 4 exposure dither pattern, for a total integration time of 843.612s per sub-band, covering the full wavelength range of the MRS. 
Combined with the NIRSpec observation, this will provide a continuous spectrum over the entire JWST wavelength range.
The host star of 2M1207b is faint, allowing for good enough contrast for a straightforward observation \parencite{Beichman2019}.

\begin{table}[t]
	\centering
	\begin{tabular}{l|ccc}
		\toprule
		\textbf{Parameter} & \textbf{VHS-1256b} & \textbf{2M1207b} & \textbf{WISE 0855}\\
		\midrule
		ObsDate & 0.0 & 0.0 & 0.0\\
		Path & SHORT/LONG & SHORT/LONG & SHORT/LONG\\
		Dither & 2 point & 4-point & None\\
		Disperser & ALL & ALL & ALL\\
		Detector & SW/LW & SW/LW & SW/LW\\
		MRS Mode & SLOW & SLOW & FAST\\
		Exposures & 1 & 1 & 1\\
		Integrations & 1 & 1 & 1\\
		Groups & 30 & 76 & 180\\
		Cosmic Rays & None & None & None\\
		\bottomrule
	\end{tabular}
	\caption[Observation Parameters]{Observing parameters for each selected target. Observation parameters are based on JWST proposals, and set in order to cover channels 1 through 3. For the disperser, ALL implies running a simulation for each of the SHORT/MEDIUM/LONG sub-bands. A total of 6 simulations are necessary to cover the entire wavelength range. Cosmic rays are turned off due to issues with MIRISIM.}
	\label{tab:obsparams}
\end{table}
\subsubsection{WISE 0855-0714}
Although it is a Y-type brown dwarf, WISE 0855 is the most similar known object to Jupiter outside our solar system that has been directly observed \parencite{Luhman2014}. 
At 250K, WISE 0855 is very faint, with an H-band magnitude of 25, but its proximity at 2pc makes it an ideal target for characterization.
The JWST GTO Program 1230 will observe WISE 0855 using NIRCam, NIRSpec and the MIRI MRS \parencite{Oliveira2019}.
It will use a FAST readout, with 180 groups per integration, and one integration per exposure for a total of 999 s of integration time for each sub-band. No dithering will be used.

As a cold object, WISE 0855 provides the best known extra-solar template for older planetary mass objects.
With the improved sensitivity and long wavelength coverage of JWST, it is hoped that more low mass and colder exoplanets may be directly imaged.
Understanding the atmosphere of WISE 0855 will provide a great deal of insight for the challenges of such exoplanetary atmospheres. 
Clouds are suspected to be present \parencite{Faherty2018}, a feature which will be better understood using mid infrared observations.

\begin{table}[t]
	\centering
	\begin{tabular}{l|ccc}
		\toprule
		\textbf{Parameter} & \textbf{VHS-1256b} & \textbf{2M1207b} & \textbf{WISE 0855}\\
		\midrule
		Radius [R$_Jup$] & 1.29 & 1.5 & 1.17\\
		Distance [pc] & 12.7 & 52.4 & 2.23\\
		$\log g$ & 4.25 & 3.2 & 4\\
		$T_{int}$ [K]& 900 & 1600 & 250\\
		$T_{equ}$ [K]& 3.4 & 10 & 3.4\\
		$\kappa_{IR}$ & 0.01 & 0.01 & 0.01\\
		$\gamma$ & 0.3 & 0.4 & 0.3\\
		\midrule
		\multicolumn{4}{c}{\textbf{Abundances}}\\
		\midrule
		H$_{2}$ & 0.898 & 0.74 & 0.73\\
		He & 0.102 & 0.24 & 0.25\\
		H$_{2}$O & 1$\times10^{-3}$ & 5$\times10^{-3}$ & 5$\times10^{-4}$\\
		CO & 1$\times10^{-7}$& 1$\times10^{-2}$ & 1$\times10^{-15}$\\
		CO$_{2}$ & 1$\times10^{-5}$& 1$\times10^{-3}$ & 1$\times10^{-14}$\\
		CH$_{4}$ & 3$\times10^{-3}$& 1$\times10^{-6}$ & 3$\times10^{-4}$\\
		NH$_{3}$ & 1$\times10^{-5}$& 1$\times10^{-7}$ & 3$\times10^{-3}$\\
		C$_{2}$H$_{2}$ & 1$\times10^{-8}$& 1$\times10^{-9}$& \ldots \\
		HCN & 1$\times10^{-10}$ & 1$\times10^{-9}$ & 1$\times10^{-9}$ \\
		TiO & \ldots & 5$\times10^{-7}$ & \ldots \\
		SiO & 1$\times10^{-6}$ &  \ldots & \ldots \\
		\bottomrule
	\end{tabular}
	\caption[petitRADTRANS inputs.]{Input parameters to generate spectra using petitRADTRANS. High resolution line-by-line mode was used. $\kappa_{IR}$ and $\gamma$ are the infrared opacity and ratio of visible to IR opacities respectively. The values chosen for these parameters are based on \parencite{Molliere2019}. The mass-fraction abundances chosen are arbitrary values chosen to encompass a wide range of compositions and to test the ability of the retrieval code to recover small abundances. Where possible, values were chosen to qualitatively reflect known species present \parencite{Miles2018}. }
	\label{tab:inputparams}
\end{table}

\subsubsection{Science Goals}
Atmospheric retrievals are currently the best tools for characterizing the composition and structure of exoplanet atmospheres. 
Parameters such as the C/O ratio may trace the formation history of planets, and may be able to settle the debate between gravitational instability and core accretion formation models \parencite{Madhusudhan2012,Moses2013}.
From solar system observations, along with our own experience on Earth, we know atmospheres are constantly changing, and time series observations will open the door to investigation of dynamics and variability.
Understanding the composition and chemistry of these atmospheres will also provide insight into the diversity - and similarity - between these systems.
Clouds are poorly understood within our own solar system, and are certain to be present in the atmospheres of other worlds.
Perhaps the most interesting prospect is uncovering novel features that have not yet been predicted, and will open the door to new avenues of exploration.

For this work, we are primarily concerned with constraining the ability of the MRS to retrieve known input parameters. 
With simulated spectra from petitRADTRANS providing a ground truth, we can compare the results of retrievals over a range of fringing cases.
%\subsection{Atmospheric Parameters}
%% CO ratio for characterization
%\cite{Garland2019} %BD absorption
%\cite{Fegley1994} %Jupiter/Saturn atmospheres
%\cite{Tokunaga1983} %MIR atmosphere spectra


\section{Methods}
Here we will outline how we generated our input spectra, and the procedure we used to perform our atmospheric retrieval.
\subsection{Spectra Generation}
\begin{figure}[t]
	\includegraphics[width=\linewidth]{petitInputSpectra}
	\caption{High resolution spectra generated by petitRADTRANS for each of the simulated targets.}
	\label{fig:petitinput}
\end{figure}

We used petitRADTRANS in high resolution, line-by-line mode in order to calculate a spectrum that can be passed as input to MIRISIM.
All three input spectra used are shown in Fig. \ref{fig:petitinput}.
The parameters chosen for each target are given in table \ref{tab:inputparams}. 
All spectra cover a range of 4.8-18.5 micron in order to fully cover channels 1 through 3 of the MRS. 
Channel 4 is ignored due to photometric calibration issues and lack of sensitivity to faint sources.

\begin{figure}[t]
	\includegraphics[width=\linewidth,trim=4.7cm 2cm 5.7cm 1cm,clip]{VHS_retrieved_extracted}
	\includegraphics[width=\linewidth,trim=4.7cm 2cm 5.7cm 1cm,clip]{VHS_retrieved_extracted_narrow_fringe}
	\caption{The extracted 1D spectrum for VHS1256b with no noise and no fringing applied compared to the input template and the best fit model from a Channel 1 atmospheric retrieval. This spectrum demonstrates the ability of the retrieval to accurately fit the data. It also shows several significant systematic effects at 5.7 and 6.4 micron, which are present at some level in all of the extracted spectra.}
	\label{fig:vhsnofringeextractedbestfit}
\end{figure}
The spectra generated by petitRADTRANS are in terms of the emitted flux and are in units of erg cm$^{-2}$ m$^{-2}$ s$^{-1}$ Hz$^{-1}$. 
MIRISIM requires the flux incident on the detector in units of $\mu$Jy, so we convert the as 
\begin{equation}
F_{inc} [\mu\textrm{Jy}] = 10^{29}\times F_{em} \times \left(\frac{R_{pl}}{d_{pl}}\right)^{2}.
\end{equation}
The wavelength grid produced by petitRADTRANS is log-spaced, and we use the \verb|spectres| python package \parencite{Carnall2017} in order to convert to a linear spaced grid with R=12000 at 4.0 $\mu$m. 
This ensures the input spectrum will oversample the instrumental spectral resolution by a factor of at least 4 across the whole wavelength range.

While it is possible to add a background term to a spectrum using MIRISIM, we chose not to use any background in order to improve our spectral extraction after processing with the JWST pipeline, with the understanding that errors from background subtraction will be negligible in actual data.

\subsection{Atmospheric Retrieval Setup}
\begin{table}[t]
	\centering
	\begin{tabular}{lll}
		\toprule
		\textbf{Parameter} & \textbf{Prior} & \textbf{Constraints}\\
		\midrule
		$\log\gamma$ & $\mathcal{N}(0,2)$&\\
		T$_{int}$ & $\mathcal{U}(0,3500)$&\\
		R$_{pl}$ & $\mathcal{U}(0.01,2.0)$&\\
		$\kappa_{IR}$ & $\mathcal{U}(0,1)$&\\
		$\log g$ & $\mathcal{U}(1.8,4.8)$&\\
		$\log P_{0}$ & $\mathcal{U}(-6,2)$&\\
		$\ln(X_{i})$ & $\mathcal{U}(-18,0)$ & $\sum X_{i} < 1$\\
		\bottomrule		
	\end{tabular}
	\caption{Prior choices for atmospheric retrievals using the standard Guillot profile. $\mathcal{U}(a,b)$ is a uniform distribution from $a$ to $b$. $\mathcal{N}(\mu,\sigma)$ is a normal distribution. $T_{int}$ corresponds to the effective temperature of an object in K, while R$_{pl}$ is the planet radius in Jupiter radii. For free floating objects, $T_{equ}$ is fixed to 2.7K. Pressures are in bar.}
	\label{tab:priors}
\end{table}

\subsubsection{Prior choice}
The choice of priors is a consistent challenge when using a Bayesian framework.
When performing a free-retrieval, uninformative priors must be chosen to allow the data to drive the posterior distribution.
This can lead to unphysical solutions or combinations of parameters.
The alternative is to use physically motivated priors, at the risk of missing unexpected phenomena.

We based our prior selection on the choices made in \parencite{Molliere2019}.
We use an uninformative uniform distribution on the temperature parameters, as well as on the log of abundance, gravity and pressure parameters.
Uniformly drawing from the log of the parameters allows for better coverage of the very large parameter space (from $10^{-15}$ to $10^{0}$ in each of the fractional abundances).
For the opacity parameters $\gamma,\delta$ and $\alpha$ we use a normal distribution centered around the expected value. 

Trial runs showed that the posterior distributions for these parameters are not driven by the priors.
Our prior choices and the ranges over which they cover are given in table \ref{tab:priors}.
% Uniform in log space
% Gaussian/normal for some
% based on petitRadTrans paper

\subsubsection{Validation}
We ran a retrieval with all of the noise and nonlinearity sources turned off in MIRISIM in order to validate our retrieval tools and to demonstrate the results of an `ideal' correction. 
This was performed using VHS1256b as the input template, and only Channel 1 was used.
While the retrieved abundances remained, the best fit model provides a much closer match to the input spectrum.
This is shown in Fig. \ref{fig:vhsnofringeextractedbestfit}, with the best fit model plotted over the input model as well as the extracted spectrum.
The best fit model closely follows the binned input spectrum across Channels 1A and 1B, and produces a significantly improved log-evidence of $-6.99\times10^{3}$. 
In Channel 1C however, the begins to deviate from the input model.
This allows us to be reasonably confident that the retrieval is being performed accurately, and the the Bayesian evidence provides a useful goodness-of-fit metric.

\clearpage
\section{Results}
Numerous atmospheric retrievals were run in order to examine the effects of fringing on parameter retrieval, as well as to better understand how to use the MRS to study atmospheres.
We will discuss the findings for each of the three targets of interest individually before discussing the effects of fringing on the outcome of the study.
The full posterior distributions are included in appendix \ref{app:post}.
%\subsection{Model Selection}
%\begin{figure}[h]
%	%\includegraphics[content...]{imagefile}
%	\caption{Bayesian evidence for models of differing dimensionality.}
%	\label{fig:bicdim}
%\end{figure}
% Number of parameters
% Bayesian evidence

\begin{table}[t]
	\begin{scriptsize}
	\begin{tabular}{l|lcccccc}
		\toprule
		\textbf{Name} & \textbf{Live} & \textbf{Wlen [$\mu m$]} & \textbf{Fringing} & \textbf{T$_{int}$ [K]} & \textbf{R$_{pl}$ [R$_{j}$]} & \textbf{C/O in} & \textbf{C/O ret}\\
		\midrule
		\multirow{3}{*}{VHS-1256b} & 1500 & 5--7.5  & No  & $684\pm0.1$ & $1.80\pm0.0004$ & $2.95$ & $0.565\pm0.003$\\
		                           & 1500 & 5--7.5 & Yes & $833\pm1$   & $1.44\pm0.004$ & $2.95$ & $3.17\pm0.18$\\
		                           & 10000 & 5--18  & No  & $734\pm0.3$ & $1.65\pm0.001$ & $2.95$ & $0.55\pm0.03$\\
		                           \midrule
		\multirow{3}{*}{WISE0855}  & 1500 & 5--7.5  & No  & $235\pm1$ & $1.24\pm0.004$ &$0.565$& $0.554\pm0.03$\\
								   & 1500 & 5--7.5  & Yes &  &  & $0.565$ &\\
							       & 400 & 5--18   & No  & $234\pm1.15$ & $1.05\pm0.02$ & $0.565$ & $0.20\pm0.37$\\
							       \midrule
		2M1207b  & 800 & 5--7.5   & No  & $1186.9\pm0.3$ & $ 0.083\pm0.0002$ & $0.647$ & $0.87\pm0.7$\\
		\bottomrule
	\end{tabular}
	\caption{Summary of atmospheric retrievals.}
	\label{tab:atmosum}
	\end{scriptsize}
\end{table}

\subsection{VHS-1256b}
We ran a set of three retrievals for VHS-1256b.
Two of these used data only from Channel 1, either with point source fringing or no fringing applied.
The third retrieval was covered the full wavelength range from 5-18 micron, but no fringing was applied to the data.
These retrievals also varied in the number of live points used, as summarized in table \ref{tab:atmosum}.
We found that the number of live points did not significantly impact the precision with which the parameters of interest could be retrieved.
However, the sampling density was improved at the cost of an approximately linear increase in computational time with the number of live points used.

All of the retrievals significantly underestimated the internal temperature of VHS1256b and overestimated the radius, with posteriors shown in Fig. \ref{fig:postVHS_tr}.
As expected from the Stephan-Boltzmann law the temperature and planet radius are strongly anti-correlated, though both parameters are biased from their true values.
The full posteriors for the structure parameters for the Channel 1 case without fringing are shown in Fig. \ref{fig:postVHS_nuisance}.
Contours are set at the 50\%, 95\% and 99\% coverage levels for all marginal plots.
As many of these parameters are not physical in nature, their absolute values are not relevant, and the pressure-temperature profile itself provides a better figure of merit for the success of the retrieval.
Fig. \ref{fig:presVHS} shows the retrieved pressure-temperature profiles as compared to the input profile. 
The computed error on the retrieved profiles are smaller than the line width.
The input profile is computed using the standard Guillot profile, and is isothermal from about 0.1 bar due to the choice of $\kappa_{IR}$ and absorption from abundant species such as water and methane in the infrared.
This is successfully retrieved if the standard Guillot profile is used, albeit at a lower internal temperature.
An inversion is found if the modified profile is used in the retrieval, rather than simplifying to the standard profile.
The simplest physically motivated pressure-temperature profile should be used in future retrievals, so as not to identify unphysical features in the atmospheric structure.
Note that this may not be the Guillot profile and that other parameterizations should be considered.

\begin{figure}[t]
	\centering
	\includegraphics[width=0.321\linewidth]{vhs_ch1_nofringe_tr.png}
	\includegraphics[width=0.28\linewidth]{vhs_ptsrc_nocorr_ch1_tr.png}
	\includegraphics[width=0.28\linewidth,trim = 0cm 0.5cm 0cm 0cm,clip]{vhsfull10000temprad.png}
	\caption{Interior temperature and planet radius posteriors for VHS-1256b.
		As expected, these parameters are highly anti-correlated: a higher temperatures and larger radii both result in increases in the overall luminosity of the planet. The true value of both parameters (900 K, 1.29 R$_{J}$) falls outside the displayed distributions.
		\textbf{Left:} Channel 1 only, with no fringing applied. 
		\textbf{Center:} Uncorrected point-source fringing was applied. 
		\textbf{Right:} Full spectrum without fringing. }
	\label{fig:postVHS_tr}
\end{figure}
\begin{figure}
	\centering
	\includegraphics[width=0.93\linewidth]{VHS1256bTempPressInputRetrieved}
	\caption{Pressure-Temperature profile for VHS-1256b, using the CH1 spectrum with no fringing or correction applied. Using different P-T profiles for the retrieval results in significant differences from the input profile. In the case where the underlying profile is unknown, multiple profiles should be compared.}
	\label{fig:presVHS}
\end{figure}

To retrieve the atmospheric composition we retrieved abundances for 8 species for VHS-1256b: CO, H$_{2}$O, CH$_{4}$, NH$_{3}$, CO$_{2}$, H$_{2}$S C$_{2}$H$_{2}$ and TiO.
These cover the primary components of the input atmosphere, while H$_{2}$S was included to examine how the retrieval would treat a species that isn't present.
Figures \ref{fig:postVHS_abundances}, \ref{fig:postVHS_abundances_fringe} and \ref{fig:postVHS_abundances_full} show the posterior distributions for the Channel 1 without fringing, Channel 1 with fringing and full spectrum cases respectively. 
The Channel 1 case without fringing was the only retrieval to identify species at the correct abundance to within the computed margin of error.
This is due both to bias in the retrieved posteriors and the narrow width of the distributions.
However, although outside of the computed error bars, the retrievals produced order of magnitude estimates for water and methane. 
CO$_{2}$ was found to be the most abundant component of the atmosphere, making up 74\% of the overall composition in the Channel 1, no fringing case.
Only the Channel 1 case with fringing found the CO$_{2}$ to within margin of error, though the retrieved value is three orders of magnitude lower than the true value.
As it is only a trace gas, a such wide posterior distribution is expected.
The results for all of the retrievals are tabulated in table \ref{tab:retspecies}. 

\begin{figure}[t]
	\centering
	\includegraphics[width=\linewidth]{vhs_nofringe_ch1_nuisance.png}
	\caption{Posterior distributions for the nuisance parameters of VHS-1256b, no fringing case and using the standard Guillot profile, using only Channel 1. As most of these are non physical parameters, the P-T distribution provides a better metric for determining whether the atmospheric structure has been retrieved as opposed to specific parameter values.}
	\label{fig:postVHS_nuisance}
\end{figure}

We find that several species are correlated, including water and methane, water and H$_{2}$S and methane and H$_{2}$S.
Water and methane both share deep absorption features at similar wavelengths, contributing to similarities in the final emission spectrum.
Hydrogen sulfide contributes primarily between 6 and 10 micron, where broadband absorption from the other two species is strongest.
This likely led to incorrect abundances for all three species, as H$_{2}$S was found to contribute over 1\% of the atmospheric composition in all three retrieval cases.
Future retrievals should use an self-consistent equilibrium chemistry model in order to verify that such results are physically valid. 
H$_{2}$S does not cross correlate with the spectral model as shown in Chapter 3, and such methods could provide a cross-check for verifying the presence of a given molecular species.

To summarize the composition and link our results to formation processes, we computed the mass fraction C/O ratio for each of our retrievals, shown in table \ref{tab:atmosum}. 
The error on the C/O ratio was found by using the 1$\sigma$ errors on each of the retrieved abundances, and adding the relative C and O errors in quadrature.
For VHS1256b we found that we could not accurately retrieve the input C/O ratio. 
This is likely due to the large abundance of H$_{2}$S affecting the retrieved abundances for water and methane, in turn impacting the C/O ratio.
Although not explored in this work, it is possible to parameterize petitRADTRANS using the C/O ratio and metallicity as opposed to single species abundances, which will likely produce more accurate results when searching for this quantity.

With the structure and composition identified we can now compare the best fit retrieved models, as well as the emission contribution function to identify which atmospheric layers we are examining.
The log-evidences for each model presented in Fig. \ref{fig:bestfitVHS} are similar: $-8.78\times10^{4}$,$-9.96\times10^{4}$ and  $-1.09\times10^{5}$ for the no fringing, Channel 1 fringing and full spectrum retrievals respectively.
While it is difficult to compare the evidence for different data sets, we do note that the no fringing case resulted in a better fit to the data than in the fringing case. 
At short wavelengths, we find that the Channel 1 fringe-free case fits the data very well, particularly compared to the point source case.
However, both of the other retrievals attempt to fit a systematic-induced feature at 6.5 $\mu$m.
At longer wavelengths we find that this model deviates from the input spectrum, and that the full spectrum model provides a better fit beyond 10 $\mu$m.
All of the model overestimate the flux prior to the 10 $\mu$m absorption feature, possibly leading to the overestimates in abundance of species such as methane, water and carbon dioxide, all of which have absorption lines around this wavelength.
In general, we find that the Channel 1, fringe-free case provides the best fit to observed absorption features, while the full spectrum retrieval provides a better fit to the continuum. 
The model found in the fringing case displays significant deviation from the input spectrum.
The emission contribution function shown in Fig \ref{fig:VHSemcont} shows the atmospheric pressures probed at each wavelength. 
The function shown is for the full spectrum retrieval with no fringing.
We see that deeper levels of the atmosphere are the emission source between 8 and 12 micron, while strong absorbers raise the photosphere at shore wavelengths.
In the case of a cloudy atmosphere, this will provide critical information for probing particle sizes and cloud base altitudes.
Even in a clear atmosphere there is no emission from below 1 bar.

\begin{figure}[t]
	\centering
	\includegraphics[width=\linewidth]{vhs_nofringe_ch1_abundance.png}
	\caption{Posterior distributions for species abundances in VHS1256b, in the fringe-free case using CH1 data. While several species were successfully retrieved, the very high abundance of H$_{2}$S, which is not present in the atmosphere and whose absorption features fall mostly outside of CH1, demonstrates the need for wide wavelength coverage to correctly identify the presence of a given species. There also exists significant correlation between H$_{2}$O and CH$_{4}$, with some additional correlation to H$_{2}$S, which has led to incorrect abundance measurements.}
	\label{fig:postVHS_abundances}
\end{figure}
\begin{figure}
	\centering
	\includegraphics[width=\linewidth]{vhs_ptsrc_nocorr_ch1_abund.png}
	\caption{Posterior distributions for species abundances in VHS1256b, in the uncorrected point-source fringing case using CH1 data.}
	\label{fig:postVHS_abundances_fringe}
\end{figure}
\begin{figure}
	\centering
	\includegraphics[width=\linewidth]{VHS_full_species_1000}
	\caption{Posterior distributions for species abundances in VHS1256b, in the no fringing case using the full spectrum. }
	\label{fig:postVHS_abundances_full}
\end{figure}

\begin{figure}[h]
	\centering
	\includegraphics[width=\linewidth,trim=3.1cm 0cm 2.8cm 0.8cm,clip]{VHS_retrieved_comp_full_v2}
	\includegraphics[width=0.48\linewidth,trim=1.0cm 0cm 1.5cm 2.5cm,clip]{VHS_retrieved_comp_5_v2}
	\includegraphics[width=0.48\linewidth,trim=1.0cm 0cm 1.5cm 2.5cm,clip]{VHS_retrieved_comp_10_v2}
	\caption{Best fit models for VHS-1256b in three different wavelength regimes.}
	\label{fig:bestfitVHS}
\end{figure}
\begin{figure}[h]
	\includegraphics[width=\linewidth]{vhs_total_emissioncont}
	\caption{Emission contribution function for the retrieved spectrum of VHS-1256b, using the full spectrum without fringing.}
	\label{fig:VHSemcont}
\end{figure}
\clearpage
\subsection{WISE 0855}
The cold temperatures and prominent molecular features make WISE0855 an excellent candidate for a retrieval study in the mid infrared.
We performed a series of three retrievals on WISE0855, following a similar setup as for VHS1256b.
We again consider two retrievals for Channel 1, both the fringing and non-fringing case.
The third case is the non-fringing, full spectrum retrieval.
We will focus our discussion on the Channel 1, no fringing case.
The full spectrum retrieval was performed with only 400 live points, leading to less accurate parameter retrieval and highly correlated abundances.
The posteriors for the full spectrum retrieval are included in appendix \ref{app:post}.

\begin{figure}[t]
	\includegraphics[width=\linewidth]{WISE0855_nofringe_pt_env}
	\caption{Pressure-temperature profile for WISE 0855.}
	\label{fig:presWISE}
\end{figure}
With strong molecular features, the retrieval performed better for WISE0855 than for the other two targets.
Posterior widths and errors computed by the retrieval present a more realistic picture of the uncertainty in measurements of the parameters.
Consider the pressure-temperature profile in Fig. \ref{fig:presWISE}.
A modified Guillot profile was used for the Channel 1 retrieval.
From 0.01 bar to 10 bar the true P-T profile falls within the 15\%$-$85\% confidence interval.
As the pressure decreases, the profiles diverge, and the retrieval did not identify the isothermal temperature structure.
We note that the input profile is not a realistic model of a brown dwarf atmosphere, and the retrieved profile likely presents a better model of a self-luminous object.
The parameters retrieved in order to compute this profile, shown in Fig. \ref{fig:postWISEnuis}, match with expectations.
The $\alpha$ parameter is centered at $0$, thus reducing the modified Guillot profile to a standard profile.
This also makes P$_{trans}$ uninformative with no impact on the profile, and this is reflected in the uniform posterior distribution.
While the remaining parameters, including the temperature and radius, exclude the true values to within 2\% due to narrow posterior distributions, they fall within 10\% of the expected value.

\begin{figure}[h]
	\includegraphics[width=\linewidth]{wise_ch1nofringe_nuis.png}
	\caption{Nuisance parameter posterior distributions for WISE0855 in the Channel 1, no fringing case.}
	\label{fig:postWISEnuis}
\end{figure}

The composition of the atmosphere is also well retrieved.
Ammonia creates the strongest absorption features, and is retrieved to within the margin of error at a mass fraction of 0.0026$\pm0.0001$.
Other strongly present species including water and methane are found to be the next most abundant molecules.
Carbon monoxide presents a false positive detection though, as it is present only at the 10$^{-15}$ level in the atmosphere, but is measured at 10$^{-4.67}$.
A similar false positive was found using the cross correlation technique in Chapter 3.
In a cold atmosphere the CO spectrum is largely featureless, and it is not expected to present in large quantities in such an atmosphere in equilibrium.
This suggests that physically motivated species selection, or self consistent modeling is necessary to validate the presence of any given measured species.
As carbon an oxygen are equally present in CO, this did not affect the C/O ratio, which was found to be $0.554\pm0.03$, within 2\% of the true value of 0.565.
The narrow posterior distributions nevertheless exclude the true value at the 3.7$\sigma$ level.

Due to the strong ammonia features and well retrieved abundance, the best fit model follows the input spectrum remarkably well.
This is reflected in the log evidence of -1.73$\times10^{4}$.
Fig. \ref{fig:WISEemcont} shows that the depth of the emission is strongly dependent on the nitrogen absorption. 
The pressure is somewhat lower than in the case of VHS1256b, likely due to stronger overall absorption. 
The large scale features centered around 7 and 14 micron trace the continuum of the CO emission spectrum, and it is possible that this structure is due to the large retrieved abundance.
\begin{figure}[h]
	\includegraphics[width=\linewidth]{wise_ch1nofringe_abund.png}
	\caption{Abundance posterior distributions for WISE0855 in the Channel 1, no fringing case.}
	\label{fig:postWISEabund}
\end{figure}
\begin{figure}[h]
	\includegraphics[width=\linewidth]{BestfitBrownDwarf_fringe}
	\includegraphics[width=\linewidth]{BestfitWISE0855}
	\caption{Best fit model for WISE0855. No fringing was applied, and only Channel 1 data was used.}
	\label{fig:bestfitWISE}
\end{figure}
\begin{figure}[h]
	\includegraphics[width=\linewidth]{wise_total_emissioncont}
	\caption{Emission contribution function for WISE0855.}
	\label{fig:WISEemcont}
\end{figure}
\clearpage
\subsection{2M1207b}
The retrievals for 2M1207b performed the worst out of all trials.
We ran the retrieval on Channel one data without fringing.
Systematics in the spectrum led to difficulties in fitting models, while the lack of significant absorption features makes identifying species difficult.
Many of the retrieved species were very strongly correlated to each other, as well as to the infrared opacity $\kappa_{IR}$.
The temperature was underestimated at 1186 K, but the radius was even more discrepant at 0.08 R$_{J}$.
We present the posterior distribution in Fig. \ref{fig:post2M} for reference, but cannot make further conclusions from these results.
\begin{figure}[h]
	\includegraphics[width=\linewidth]{2M1207b_CH1_nofringe_corner}
	\caption{Posterior Distributions for 2M1207b.}
	\label{fig:post2M}
\end{figure}
\clearpage
\begin{landscape}
	\begin{table}[h]
		\centering
		\begin{footnotesize}
		\begin{tabular}{l|lccccccc}
			\toprule
			\textbf{Name} & \textbf{Wlen [$\mu m$]} & \textbf{Fringing} & \textbf{H$_{2}$O} & \textbf{CH$_{4}$} & \textbf{NH$_{3}$} &\textbf{CO} & \textbf{CO$_{2}$} & \textbf{H$_{2}$S}\\
			\midrule
			\multirow{4}{*}{VHS-1256b}&  Truth &     & $1\times10^{-3}$ &$3\times10^{-3}$&$1\times10^{-5}$&$1\times10^{-7}$&$1\times10^{-5}$& $0.0$\\
									  & 5--7.5 & No  &$1.02\pm0.001\times10^{-2}$& $1.86\pm0.0002\times10^{-2}$&$1.9\pm0.2\times10^{-8}$&$8\pm2\times10^{-11}$&$7.4\pm0.01\times10^{-1}$&$1.9\pm0.01\times10^{-1}$\\
									  
			 						  & 5--7.5 & Yes &$3.6\pm0.3\times10^{-3}$& $1.3\pm0.1\times10^{-2}$&$1.0\pm0.3\times10^{-5}$&$0.3\pm1.5\times10^{-8}$&$0.08\pm1\times10^{-6}$&$7.4\pm0.7\times10^{-1}$\\
			 						  
			 						  & 5--18  & No  &$9.1\pm0.6\times10^{-4}$& $7.2\pm0.2\times10^{-3}$&$0.2\pm1\times10^{-9}$&$0.4\pm1.0\times10^{-10}$&$1.9\pm0.1\times10^{-1}$&$7.9\pm0.2\times10^{-1}$\\
			\midrule
			\multirow{4}{*}{WISE0855} &  Truth &     &$4.6\times10^{-4}$&$2.6\times10^{-4}$&$2.6\times10^{-3}$&$1\times10^{-15}$&$1\times10^{-14}$& $0.0$ \\
									  & 5--7.5 & No  &$5.6\pm0.1\times10^{-4}$&$3.0\pm0.1\times10^{-4}$&$2.6\pm0.1\times10^{-3}$&$2.1\pm0.1\times10^{-5}$&$0.5\pm5\times10^{-9}$&$0.7\pm7\times10^{-8}$\\
			 						  & 5--7.5 & Yes &&&&&&\\
								      & 5--18  & No  &$2.5\pm2\times10^{-1}$&$4.6\pm0.9\times10^{-2}$&$3.7\pm4\times10^{-1}$&$0.02\pm3\times10^{-7}$&$0.01\pm5\times10^{-6}$&\ldots \\
			\midrule
			\multirow{3}{*}{2M-1207b} &  Truth &     &$5\times10^{-3}$&$1\times10^{-6}$&$1\times10^{-7}$&$1\times10^{-2}$&$1\times10^{-3}$&$0.0$\\
									  & 5--7.5 & No  &$7\pm2\times10^{-2}$&$1.6\pm0.5\times10^{-4}$&$9\pm4\times10^{-4}$&$4.5\pm3\times10^{-1}$&$0.05\pm3\times10^{-8}$&\ldots\\
			\bottomrule
		\end{tabular}
	\caption{Retrieved mass fraction abundances for a selection of retrieved species. The log of these values is the retrieved parameter, where applicable the stated error is the 1$\sigma$ error in log space. This error can result in abundances below zero. For the lower bound the log error in the posterior plots should be used.}
	\label{tab:retspecies}
	\end{footnotesize}
	\end{table}
\end{landscape}
\subsection{Fringing Comparison}
We begin by comparing the results of different fringing cases, using VHS-1256b as a template.
The first set of retrievals use only Channel 1 data from 4.9-7.5 $\mu$m, as point source fringe flats have only been generated for this wavelength range.
We compare the case of a retrieval with no fringing added to the spectrum to having an on-axis point-source derived fringe flat applied and an extended-source fringe flat used for correction.
The case of an extended source fringe flat was not used, as the correction is near ideal, and it is well represented by the no-fringing case.
The effects of residual fringe correction on the retrieval was not examined.

We found that point source fringing did not result in a significant reduction in precision or accuracy of the retrieval results.
While it may have an impact, other sources of error led to far greater variation in retrieval results.
Parameters were not retrieved to within the computed margin of error.
For the abundance measurements for VHS-1256b, the point source fringing case provided more accurate estimates than in the no fringing case, as shown in table \ref{tab:retspecies}.
The retrieved temperature and radius of VHS-1256b were also less discrepant in the fringing case than in the non fringing case.
However, we cannot attribute the improved accuracy of the retrieval to the addition of fringing.
These results were inconsistent across multiple trials, and the posteriors presented in Fig. \ref{fig:postVHS_abundances_fringe} represent the best retrieved parameters. 

Differences in processing, noise, and choices of retrieval hyperparameters will all significantly impact the results.
We cannot conclude that atmospheric retrievals provide a good metric by which to measure the effects of fringing in the MIRI MRS.
At the present, fringing is also not the limiting factor in improving the precision of atmospheric retrievals.

\subsection{Discussion}
The three targets chosen represent a relatively diverse selection of sub-stellar objects, yet there are similarities across all of the retrievals.
Temperatures are uniformly underestimated and the radii overestimated, regardless of the choice of pressure-temperature profile parameterization.
There are correlations between these parameters and the abundances of several species, including water and methane.
These abundances are also not retrieved to within the margin of error.
The mechanism for these systematic biases is unclear, and should be explored through a series of retrievals examining the impact of individual species on the temperature and radius.
Additional structure parameterizations should be explored in order to determine if the temperature and radius offsets are due to the chosen model, or to sampling effects.
A similar effect has been noted for transmission spectroscopy in an upcoming study from \parencite{MacDonald2020}, though their mechanism is not necessarily applicable to self-luminous objects.

Posterior widths and errors are systematically underestimated. 
Given that most parameters can be retrieved to within 10\% at best, the narrow distributions does not accurately represent the variance in the retrieved parameters.
This is likely due to the small ($\approx$1\%) errors associated with the extracted JWST data, as well as the very large number of data points.
It does not reflect the large systematics that dominate the variance in the spectrum.
True error propagation through the JWST pipeline is unlikely to resolve this issue, as the relative errors associated with each spectral point will be of the same order as the photometric errors used in this work.
In order to justify such small errors, calibration must be improved such that systematic effects are at the same level or lower as the various noise sources impacting the data.

False detections of species are another common feature to our retrievals.
These tend to be species without significant absorption features, and are generally correlated with atmospheric profile parameters, as well as the abundances of common molecules.
These species are also not expected at the temperatures and pressures associated with the objects that they are observed in.
Due to these false detections and their potential impact on the abundances of other species, we found that the C/O ratio could not be accurately retrieved using retrievals on species abundance.
Model parameterization using the C/O ratio and metallicity may be necessary to measure this quantity.
For VHS-1256b, the use of a cross correlation on the retrieved species would indicate that the species is not present in the spectrum.
A self-consistent model would provide an additional check on such species, but constrains the possible atmospheres that can be measured.
Presently disequilibrium models are too slow to be used in a retrieval, but may be necessary to validate the presence of retrieved species.
The effect of individual species on the Bayesian evidence should also be examined in order to justify the inclusion of the species in the retrieval.
Any single retrieval is insufficient for claiming an understanding of the atmospheric structure and composition.
Retrievals using different models must be compared, and the results must be checked to ensure that they are physically valid.

We must also acknowledge that the results presented here will not reflect the observed spectra from these particular JWST targets.
Clouds are certain to impact the spectra, and add an additional layer of complication and uncertainty to the retrieval process.
These objects will vary in time, and will have 3D features that impact observations.
Therefore we caution that this work is very much an idealization, and represents somewhat of a best case scenario for the retrieval process for the MIRI MRS.
 
% This all ignores inherent issues that may be present in petitRADTRANS. However, the purpose of this work is not to realistically model atmospheres, but rather to show the extent to which current tools can retrieve input parameters. Other works such as \parencite{Barstow2020} have compared various retrieval and modelling tools, finding that while generally similar results are obtained, there are scenarios where different tools lead to different posterior distributions. 