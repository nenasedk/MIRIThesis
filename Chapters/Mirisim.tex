\chapter{Fringing Effects in MIRISIM}
\section{MIRISIM}
The MIRI instrument has been modeled in python as a program known as MIRISM. 
This program takes in an astronomical 'scene' along with some configuration parameters to output a detector data product, similar to what will be produced by the actual instrument.
This is relatively full-featured simulator, modelling the instrumental PSF, various noise sources and distortion maps, among other effects.
While MIRISIM is functional for all of the MIRI sub-instruments, this report will only deal with the Medium-Resolution Spectrometer (MRS) sub instrument, described in section \ref{sec:mrs}.
The goal of this section is to describe the implementation and testing of an updated optical model of the `fringing' effect - an optical effect caused by thin film interference from the multiple layers of the detector.
\subsection{Architecture}
SCENE - SEDs
SIMULATOR
PYSPECSIM
\subsection{Data Products}
\subsection{Instrumental effects}
\subsection{Fringing}
A key effect on spectral data is fringing, described in \cite{ref:Argyriou2018}. 
MIRI uses a total of three Si:As impurity band conduction detector arrays, two of which are used by the MRS. 
These detectors consist of 7 layers, listed in table \ref{tab:layers} and illustrated in Fig. \ref{fig:layers}.

\begin{figure}
	%\includegraphics[width=\linewidth]{}
	\caption{\label{fig:layers}}
\end{figure}
\begin{table}
	\begin{tabular}{llll}
	\end{tabular}
	\caption{\label{tab:layers}}
\end{table}

Thin film interference occurs when light is coherently reflected at the boundary between two layers and interferes with the incident light.
This is the principle on which Fabry-P\'{e}rot interferometers function.
As we wish to determine the effect of fringing on the amplitude of the signal received by the detector, we are effectively interested in the transmittance of a series of Fabry--P\'{e}rot interferometers. 
Assuming an ideal plane-parallel optical cavity with a reflectance R at both boundaries, thickness D, and an angle $\theta$ at which the light travels within the cavity, we can compute the transmittance as:
\begin{equation}\label{eqn:trans}
T_{c} = \frac{1}{1+\frac{4R}{\left(1-R\right)^{2}}\sin^{2}\left(\frac{\delta}{2}\right)}
\end{equation}
Where the phase $\delta$ at half a wavelength ($\phi = \pi$), with wavenumber $\sigma$ is:
\begin{equation}\label{eqn:phase}
\delta = 4\pi\sigma D \cos\theta - (\phi - \pi)
\end{equation}

If all of the parameters were known, this would be sufficient to numerically solve for the fringing pattern within MIRI. 
Unfortunately, uncertainties in the thickness in the detector layers, variations in the layer deposition thickness, and the uncertainty of transmittance and reflectance of the materials used at cryogenic temperatures prevents the implementation of such a numerical model.

Instead, we turn to calibration data taken to characterize the fringing pattern.

 ***DESCRIBE CURRENT MODEL - GENERIC FRINGING***
 
However, due to the dependance of fringing on the incident angle of the light, a single model of fringing is insufficient to describe the full effect. 
Therefore, we use data taken in XXXXXXXXXX at various points across the detector and quantify how this changes the extracted spectra after processing in the JWST pipeline.

 ***DESCRIBE HOW THE DATA WAS TAKEN HERE***.
 - Problems with point vs extended sources
 - multiple collection runs

\begin{figure}
	%\includegraphics[width=\linewidth]{}
	\caption{\label{fig:fringeflat}}
\end{figure}
Ultimately this data collection produced a series of 'fringe-flats' of an almost point like at various position across the detector and in each channel.
We implemented a new routine into the pySpecSim portion of MIRISIM to read in the location of point sources within a scene, and apply the correct position dependent fringe flat. 
This implementation comes with several caveats: namely that the fringing model is not yet fully developed, so it can only be considered accurate for point sources located at the same ($\alpha,\beta$) location as the source used to produce the fringe flat. Additionally, the source used to generate the data is not a true point source, nor are there fringe flats produced for the full MRS wavelength range.
We stress that the goal of this testing is to demonstrate the significance of this effect to justify the need for a more complete model along with additional calibration data to constrain the detector layer parameters.
\subsubsection{FM Data}
\subsubsection{CV Data}
\subsection{Implementation}
\subsection{JWST Pipeline}
\subsubsection{Stage 1 Processing}
\subsubsection{Stage 2 Processing}
\subsubsection{Fringing correction}
\subsubsection{Aperture Photometry}
\subsection{Cross-Correlation}
To quantify the similarity of the spectrum output by the JWST pipeline to the input into MIRISIM, we rely on the technique of cross correlation \cite{}.
For two arbitrary, complex-valued functions $f(t)$ and $g(t)$, we can compute the cross correlation as as function of the shift $\tau$ between the functions (typically in time or velocity space):
\begin{equation}\label{eqn:crosscorr}
\left(f \star g\right)(\tau) \equiv \int_{-\infty}^{\infty}f^{*}(t)g(t + \tau)dt
\end{equation}
Our signals of interest are astrophysical spectra, measured in a finite number of discrete wavelength bins. For such a signal with $M$ bins:
\begin{equation}\label{eqn:discretecorr}
\left(f \star g\right)[n] \equiv \sum_{m=0}^{M}f^{*}[m]g[m + n]
\end{equation}

Care must be taken when cross-correlating signals, as differences in normalization can result in changes in the correlation coefficient. 
Our procedure takes in two spectra. 
The first is an emission spectrum produced by the petitRadTrans program \cite{Molliere2019}, which provides our forward model with which we compare our data spectrum.
Our data is the result of passing the template spectrum through MIRISIM, and extracting it from the resulting detector image using the JWST pipeline.
We then rebin the high-resolution input spectrum to the same wavelength bins as the data spectrum, using the \verb|spectres| package \cite{}.
Prior to normalization, we remove any outliers from the spectrum (due to binning errors or instrumental effects) by setting any data points separater by more than 15 standard deviations from the mean to the median value of the spectrum.
For each spectrum, we subtract the minimum value to remove any offset in the spectrum, and divide by the maximum value to restrict the range to [0,1]. 
We then use apply a Savitzky-Golay filter with a window of 1/4 the length of the spectrum and a polynomial order of 3, which we then subtract from the unfiltered spectrum. 
This removes the continuum emission from the spectrum, and centers it around 0.
We then renormalize the spectrum by dividing by the maximum absolute value such that the range is in [-1,1]. 
The cross correlation between the forward model and itself is computed, excluding the region of interest around 0 offset. 
This `autocorrelation' is subtracted from the cross correlation between the forward model and the data spectrum in order to remove secondary peaks.
Finally, we normalize the cross correlation by the standard deviation of the cross correlation (excluding the central peak), giving an output measured as a signal to noise ratio.

\subsection{Residual Statistics}
In addition to computing the cross correlation between the forward model and the data spectrum, we also examine the residuals between the two spectra.
Here we can see any unexpected variations between the two (periodic signals, offsets or other features).
We can also examine a histogram of the residuals, normalized by the standard deviation of the data spectrum.
This provides us with a  distribution which should have a mean of 0 and unit width if the data are unbiased and share a distribution with the true input spectrum.
\section{Fringing Results}
1. A stronger input signal results in a stronger correlation.
2. Fringing does NOT necessarily degrade the cross correlation SNR, but rather increases it. The scale of this increase seems to depend on the absolute magnitude of the correlation (ie, a larger increase at higher SNR)
3. The residuals from subtracting the template from the data has structure.
4. If the residuals are histogrammed (and normalized by the standard deviation of the data), the width of the distribution may correspond to the cross correlation SNR (wider distribution = lower SNR)
5. Only when strongly increasing the fringing effect does the SNR decrease.
6. Correcting for fringing using the standard JWST fringe map decreases the SNR when compared to the case of fringing with no correction, but is typically still above the no-fringing case.
7. The JWST correction performs worse in the off axis case, as the fringe pattern begins to vary more when compared to the CV fringing model.
\subsection{Effects of fringing on spectral extraction}